\documentclass[10pt, a4, oneside]{article}
\usepackage[utf8]{inputenc}
\usepackage[spanish]{babel}
\usepackage{amsmath}
\usepackage{amsfonts}
\usepackage{amssymb}
\usepackage{graphicx}
\usepackage[left=1.5cm,right=1.5cm,top=1.5cm,bottom=1.5cm]{geometry}
\usepackage{setspace}
\parskip=2mm

\title{Notebook SWERC 2014}
\author{Anders Jonsson, Javier Segovia}
\date{The Binary Three: Sergio Almendros, Daniel Furelos, Adrià Garriga}
\makeindex

\begin{document}
\maketitle

Some parts in the sections marked with [STU] are taken from Stanford
University's ACM Team notebook, available online at
\verb-http://stanford.edu/~liszt90/acm/notebook.html-

\tableofcontents

\section{Trivia}
\subsection{Precisión cout}
\begin{verbatim}
cout.setf(ios::fixed);
cout.precision(8);
\end{verbatim}
\subsection{Librerías}
\begin{tabular}{|r|l|r|l|r|l|}
\hline
\verb-algorithm- & \verb-sort-, \verb-(next|prev)_permutation- & \verb-map- & \verb-map<S,T>- &
\verb-cfloat- & \verb-DBL_MAX- \\
\hline
\verb-queue- & \verb-priority_queue- & \verb-cmath- & \verb-pow-, \verb-sqrt- &
\verb-set- & \verb-set<S>- \\
\hline
\verb-iostream- & \verb-cin-, \verb-cout- & \verb-string- & \verb-string- &
\verb-iomanip- & \verb-setprecision- \\
\hline
\verb-utility- & \verb-pair<S,T>- & \verb-list- & \verb-list<T>- &
\verb-vector- & \verb-vector<T>- \\
\hline
\verb-cstdlib- & \verb-abs-, \verb-rand- & \verb-sstream- &
\verb-(i|o)stringstream- & \verb-iomanip- & \verb-setw-\\
\hline
\end{tabular}

\subsection{Macros}
\begin{verbatim}
#define X first
#define Y second
#define LI long long
#define MP make_pair
#define PB push_back
#define SZ size()
#define SQ(a) ((a)*(a))
#define MAX(a,b) ((a)>(b)?(a):(b))
#define MIN(a,b) ((a)<(b)?(a):(b))
#define FOR(i,x,y) for(int i=(int)x; i<(int)y; i++)
#define RFOR(i,x,y) for(int i=(int)x; i>(int)y; i--)
#define SORT(a) sort(a.begin(), a.end())
#define RSORT(a) sort(a.rbegin(), a.rend())
#define IN(a,pos,c) insert(a.begin()+pos,1,c)
#define DEL(a,pos,cant) erase(a.begin()+pos,cant)
\end{verbatim}

\section{Números}
\subsection{Primos}
Guardar todos los primos entre $0$ y $10010$ en \verb-v-. Con esto podremos
probar primos hasta $10^8$.
\begin{verbatim}
int k = 0, v[10000];
v[k++] = 2; 
for (int i = 3; i <= 10010; i += 2) {
  bool b = true;
  for (int j = 0; b && v[j]*v[j] <= i; j++)
    b = i%v[j] > 0;
  if (b) v[k++] = i;
}
\end{verbatim}
\subsection{Máximo Común Divisor (Euclido)}
\begin{verbatim}
int gcd(int a, int b) {
  if (a < b) return gcd(b, a);
  else if (a%b == 0) return b;
  else return gcd(b, a%b);
}
\end{verbatim}
\subsection{Mínimo Común Múltiple}
\verb-gcd(a,b)*lcm(a,b) = a*b-
\subsection{[STU] Módulo inverso $ax (mod M) = 1$, módulo positivo}
\begin{verbatim}
// return a % b (positive value)
int mod(int a, int b) {
  return ((a%b)+b)%b;
}

// returns d = gcd(a,b); finds x,y such that d = ax + by
int extended_euclid(int a, int b, int &x, int &y) {  
  int xx = y = 0;
  int yy = x = 1;
  while (b) {
    int q = a/b;
    int t = b; b = a%b; a = t;
    t = xx; xx = x-q*xx; x = t;
    t = yy; yy = y-q*yy; y = t;
  }
  return a;
}
int mod_inverse(int a, int n) {
  int x, y;
  int d = extended_euclid(a, n, x, y);
  if (d > 1) return -1;
  return mod(x,n);
}
\end{verbatim}
\subsection{Ejemplo next\_permutation}
\begin{verbatim}
int arr[] = {1, 2, 3};
sort(arr, arr+3); //importante
while(next_permutation(arr,arr+3)) ;
\end{verbatim}

\subsection{Coeficiente Binomial (n sobre k)}
\[ \left( {r \choose k} \right) = {r+k-1 \choose k} \]
\begin{verbatim}
int bcoeff(int n, int k) {
  if(k > n-k) k = n-k;
  if(k == 0 || n <= 1) return 1;
  return bcoeff(n-1, k) + bcoeff(n-1, k-1)
}
\end{verbatim}

\subsection{Combinatorio de Catalán}
\begin{verbatim}
unsigned long long v[34];
void catalan() {
  v[0] = 1;
  for (int i = 1; i <= 33; i++)
    v[i] = ((2*i - 1) * 2 * v[i-1]) / (i+1);
}
\end{verbatim}

\subsection{Clase Números Grandes}
Needs \verb-vector-, \verb-string-, \verb-iostream-, \verb-cassert-, \verb-iomanip-.
\begin{verbatim}
#define BASE 1000000000

struct big {
  vector<int> V;
  big(): V(1, 0) {}
  big(int n): V(1, n) {assert(n < BASE);}
  big(const big &b): V(b.V) {}

  bool operator==(const big &b) const { return V==b.V; }
  int &operator[](int i) { return V[i]; }
  int operator[](int i) const { return V[i]; }
  int size() const { return V.SZ; }
  void resize(int i) { V.resize(i); }

  bool operator<(const big &b) const {
    for (int i = b.SZ-1; SZ == b.SZ && i >= 0; i--)
      if (V[i] == b[i]) continue;
      else return (V[i] < b[i]);
    return (SZ < b.SZ);
  }

  void add_digit(int l) {
    if (l > 0) V.PB(l);
  }
};

big suma(const big &a, const big &b, int k) {
  LI l = 0;
  int size = MAX(a.SZ, b.SZ+k);
  big c;  c.resize(size);
  for (int i = 0; i < size; ++i) {
    l += i < a.SZ ? a[i] : 0;
    l += (k <= i && i < k + b.SZ) ? b[i-k] : 0;
    c[i] = l%BASE;
    l /= BASE;
  }
  c.add_digit(int(l));
  return c;
}

big operator+(const big &a, const big &b) {
  return suma(a, b, 0);
}
big operator+(const big &a, int b) {return a+big(b);}
big operator+(int b, const big &a) {return a+big(b);}


big operator-(const big &a, const big &b) {
  assert(b < a || a == b);
  LI l = 0, m = 0;
  big c;  c.resize(a.SZ);
  for (int i = 0; i < a.SZ; ++i) {
    l += a[i];
    l -= i < b.SZ ? b[i] + m : m;
    if (l < 0) { l += BASE; m = 1; }
    else m = 0;
    c[i] = l%BASE;
    l /= BASE;
  }
  if (c[c.SZ-1] == 0 && c.SZ > 1) c.resize(c.SZ-1);
  return c;
}
big operator-(const big &a, int b) {return a-big(b);}

big operator*(const big &a, int b) {
  if (b == 0) return big(0);
  big c;  c.resize(a.SZ);
  LI l = 0;
  for (int i = 0; i < a.SZ; ++i) {
    l += (LI)b*a[i];
    c[i] = l%BASE;
    l /= BASE;
  }
  c.add_digit(int(l));
  return c;
}
big operator*(int b, const big &a) {return a*b;}
big operator*(const big &a, const big &b) {
  big res;
  for (int i = 0; i < b.SZ; ++i)
    res = suma(res, a*b[i], i);
  return res;
}

void divmod(const big &a, int b, big &div, int &mod) {
  div.resize(a.SZ);
  LI l = 0;
  for (int i = a.SZ-1; i >= 0; --i) {
    l *= BASE;
    l += a[i];
    div[i] = l/b;
    l %= b;
  }
  if (div[div.SZ-1] == 0 && div.SZ > 1) div.resize(div.SZ-1);
  mod=int(l);
}

big operator/(const big &a, int b) {
  big div;  int mod;
  divmod(a, b, div, mod);
  return div;
}

int operator%(const big &a, int b) {
  big div;  int mod;
  divmod(a, b, div, mod);
  return mod;
}

istream &operator>>(istream &is, big &b) {
  string s;
  if (is >> s) {
    b.resize((s.SZ - 1)/9 + 1);
    for (int n = s.SZ, k = 0; n > 0; n -= 9, k++) {
      b[k] = 0;
      for (int i = MAX(n-9, 0); i < n; i++)
      b[k] = 10*b[k] + s[i]-'0';
    }
  }
  return is;
}

ostream &operator<<(ostream &os, const big &b) {
  os << b[b.SZ-1];
  for (int k = b.SZ-2; k >= 0; k--)
    os << setw(9) << setfill('0') << b[k];
  return os;
}
\end{verbatim}


\section{Matrices}
Necesita \verb-cstring-.
\begin{verbatim}
#define SIZE 15    // tamaño de matriz cuadrada
#define MOD 10007  // módulo de la multiplicación
struct matriz {
  int v[SIZE][SIZE];
  matriz() { memset(v, 0, SIZE*SIZE*sizeof(int)); } // matriz de 0's
  matriz(int x) {      // matriz con x's en la diagonal
    memset(v, 0, SIZE*SIZE*sizeof(int));
    for (int i = 0; i < SIZE; i++) v[i][i] = x;
  }
  // multiplicación de matrices módulo MOD
  matriz operator*(matriz &m) {
    matriz n;
    for (int i = 0; i < SIZE; i++)
      for (int j = 0; j < SIZE; j++)
    for (int k = 0; k < SIZE; k++)
      n.v[i][j] = (n.v[i][j] + v[i][k]*m.v[k][j])%MOD;
    return n;
  }
};
\end{verbatim}

\subsection{Exponenciación (simulaciones)}
Funciona con \verb-T = matriz- y \verb-U = big-;
\begin{verbatim}
template <typename T,typename U> T expo(T &t, U n) {
  if (n == U(0)) return T(1);
  else {
    T u = expo(t, n/2);
    if (n%2 > 0) return u*u*t; else return u*u;
  }
}
\end{verbatim}

\section{Grafos}
\subsection{[STU] Camino Euleriano (pasar por todas las aristas)}
\begin{verbatim}
struct Edge;
typedef list<Edge>::iterator iter;

struct Edge
{
  int next_vertex;
  iter reverse_edge;

  Edge(int next_vertex)
    :next_vertex(next_vertex)
    { }
};

const int max_vertices = 3000;
int num_vertices;
list<Edge> adj[max_vertices];    // adjacency list

vector<int> path;

void find_path(int v) {
  while(adj[v].size() > 0) {
    int vn = adj[v].front().next_vertex;
    adj[vn].erase(adj[v].front().reverse_edge);
    adj[v].pop_front();
    find_path(vn);
  }
  path.push_back(v);
}

void add_edge(int a, int b) {
  adj[a].push_front(Edge(b));
  iter ita = adj[a].begin();
  adj[b].push_front(Edge(a));
  iter itb = adj[b].begin();
  ita->reverse_edge = itb;
  itb->reverse_edge = ita;
}
\end{verbatim}

\subsection{Dijkstra (con listas de adyacencia)}
\begin{verbatim}
typedef int V;          // tipo de costes
typedef pair<V,int> P;  // par de (coste,nodo)
typedef set<P> S;       // conjunto de pares

int N;                  // numero de nodos
vector<P> A[10001];     // listas adyacencia (coste,nodo)
int prec[201]; // predecesores

V dijkstra(int s, int t) {
  S m;                          // cola de prioridad
  vector<V> z(N, 1000000000);   // distancias iniciales
  z[s] = 0;                     // distancia a s es 0
  m.insert(MP(0, s));           // insertar (0,s) en m
  while (m.SZ > 0) {
    P p = *m.begin();   // p=(coste,nodo) con menor coste
    m.erase(m.begin()); // elimina este par de m
    if (p.Y == t) return p.X; // cuando nodo es t, acaba
    // para cada nodo adjacente al nodo p.Y
    for (int i = 0; i < (int)A[p.Y].SZ; i++) {
      // q = (coste hasta nodo adjacente, nodo adjacente)
      P q(p.X + A[p.Y][i].X, A[p.Y][i].Y);
      // si q.X es la menor distancia hasta q.Y
      if (q.X < z[q.Y]) {
    m.erase(MP(z[q.Y], q.Y)); // borrar anterior
    m.insert(q);              // insertar q
    z[q.Y] = q.X;             // actualizar distancia
    prec[q.Y] = p.Y;          // actualizar predecesores
      }
    }
  }
  return -1;
}

int main() {
  N = 6;             // solucion 0-1-2-4-3-5, coste 11
  A[0].PB(MP(2, 1)); // arista (0, 1) con coste 2
  A[0].PB(MP(5, 2)); // arista (0, 2) con coste 5
  A[1].PB(MP(2, 2)); // arista (1, 2) con coste 2
  A[1].PB(MP(7, 3)); // arista (1, 3) con coste 7
  A[2].PB(MP(2, 4)); // arista (2, 4) con coste 2
  A[3].PB(MP(3, 5)); // arista (3, 5) con coste 3
  A[4].PB(MP(2, 3)); // arista (4, 3) con coste 2
  A[4].PB(MP(8, 5)); // arista (4, 5) con coste 8
  cout << dijkstra(0, 5) << endl;
}
\end{verbatim}
Después podemos ir siguiendo el array de predecesores después para ver el
camino mínimo.

\subsection{Kruskal (Minimum Spanning Tree)}
$O(E log V)$
\begin{verbatim}
typedef vector<pair<int,pair<int,int> > > V;

vector< pair<long, int> > K[2000]; // kruskal tree
int N, mf[2000]; // numero de nodos N <= 2000
V v;             // vector de aristas
                 // (coste, (nodo1, nodo2))

int set(int n) { // conjunto conexo de n
  if (mf[n] == n) return n;
  else mf[n] = set(mf[n]); return mf[n];
}

int kruskal() {
  int a, b, sum = 0;
  sort(v.begin(), v.end());
  for (int i = 0; i < N; i++)
    mf[i] = i; // inicializar conjuntos conexos

  for (int i = 0; i < (int)v.SZ; i++) {
    a = set(v[i].Y.X), b = set(v[i].Y.Y);
    if (a != b) { // si conjuntos son diferentes
      mf[b] = a;  // unificar los conjuntos
      sum += v[i].X; // agregar coste de arista
      //estas dos lineas guardan el arbol visitados
      K[v[i].Y.X].PB(MP(v[i].X, v[i].Y.Y));
      K[v[i].Y.Y].PB(MP(v[i].X, v[i].Y.X));
    }
  }
  return sum;
}

int main() {
  N = 5;               // solucion 13:
                       // (0,3),(1,2),(2,3),(3,4)
  v.PB(MP(4,MP(0,1))); // arista (0,1) coste 4
  v.PB(MP(4,MP(0,2))); // arista (0,2) coste 4
  v.PB(MP(3,MP(0,3))); // arista (0,3) coste 3
  v.PB(MP(6,MP(0,4))); // arista (0,4) coste 6
  v.PB(MP(3,MP(1,2))); // arista (1,2) coste 3
  v.PB(MP(7,MP(1,4))); // arista (1,4) coste 7
  v.PB(MP(2,MP(2,3))); // arista (2,3) coste 2
  v.PB(MP(5,MP(3,4))); // arista (3,4) coste 5
  cout << kruskal() << endl;
}
\end{verbatim}

\subsection{Maxflow}
\begin{verbatim}
#define N 10000 // numero maximo de nodos

typedef pair<int,int> P;

int n;          // numero de nodos
vector<P> A[N]; // listas adyacencia (vecino,capacidad)
vector<P> C[N]; // listas adyacencia aumentadas

int BFS(int s, int t) {
  int k = 0, l[N], q[N], r[N], v[N];
  for (int i = 0; i < n; i++) v[i] = -1;
  v[s] = 1000000000;  l[k++] = s;
  for (int i = 0; i < k; i++)
    for (int j = 0; j < (int)C[l[i]].size(); j++) {
      P p = C[l[i]][j];
      if (p.Y > 0 && v[p.X] < 0) {
    l[k++] = p.X;
    v[p.X] = MIN(v[l[i]], p.Y);
    q[p.X] = l[i];  r[p.X] = j;
      }
    }
  if (v[t] < 0) return 0;
  for (int a = t; a != s; a = q[a]) {
    C[q[a]][r[a]].Y -= v[t];
    C[a][bb(MP(q[a], 0), C[a])].Y += v[t];
  }
  return v[t];
}

int maxflow(int s, int t) { // flujo maximo entre s y t
  for (int i = 0; i < n; i++)
    sort(A[i].begin(), A[i].end());
  for (int i = 0; i < n; i++)
    for (int j = 0; j < (int)A[i].SZ; j++) {
      int x = A[i][j].X, y = bb(MP(i, 0), A[x]);
      if (y == A[x].SZ || i < A[x][y].X)
    C[x].PB(MP(i, 0));
    }
  for (int i = 0; i < n; i++) {
    C[i].insert(C[i].end(), A[i].begin(), A[i].end());
    sort(C[i].begin(), C[i].end());
  }
  int sum = 0;
  for (int inc = 1; inc > 0; sum += inc = BFS(s, t));
  return sum;
}
\end{verbatim}

\subsection{Tarjan (Strongly connected components)}
$O(|V| + |E|)$
\begin{verbatim}
int index, ct;
vector<bool> I;
vector<int> D, L, S;
vector<vector<int> > V; // listas de adyacencia

void tarjan (unsigned n) {
  D[n] = L[n] = index++;
  S.push_back(n);
  I[n] = true;
  for (unsigned i = 0; i < V[n].size(); ++i) {
    if (D[V[n][i]] < 0) {
      tarjan(V[n][i]);
      L[n] = MIN(L[n], L[V[n][i]]);
    }
    else if (I[V[n][i]])
      L[n] = MIN(L[n], D[V[n][i]]);
  }
  if (D[n] == L[n]) {
    ++ct;
    // todos los nodos eliminados de S pertenecen al mismo scc
    while (S[S.size() - 1] != n) {
      I[S.back()] = false;
      S.pop_back();
    }
    I[n] = false;
    S.pop_back();
  }
}

void scc() {
  index = ct = 0;
  I = vector<bool>(V.size(), false);
  D = vector<int>(V.size(), -1);
  L = vector<int>(V.size());
  S.clear();
  for (unsigned n = 1; n <= V.size(); ++n)
    if (D[n] < 0)
      tarjan(n);
  // ct = numero total de scc
}
\end{verbatim}

\subsection{Bron-Kerbosch (máximo conjunto independiente)}
Conjunto independiente = no hay dos nodos que sean adyacentes.
\begin{verbatim}
#define U unsigned int
typedef vector<short int> V;

vector<vector<U> > graf; // vertices/aristas del grafo
U numv, kmax; // # conjuntos/tamano grupo independiente

int evalua(V &vec) {
  for (int n = 0; n < vec.size(); n++)
    if (vec[n] == 1) return n;
  return -1;
}

void Bron_i_Kerbosch() {
  vector<U> v;
  U i, j, aux, k = 0, bandera = 2;
  vector<V> I, Ve, Va;
  I.PB(V()); Ve.PB(V()); Va.PB(V());
  for (i = 0; i < numv; i++) {
    I[0].PB(0);  // conjunto vacio
    Ve[0].PB(0); // conjunto vacio
    Va[0].PB(1); // contiene todos
  } 
  while(true) {
    switch(bandera) {
    case 2: // paso 2
      v.PB(evalua(Va[k]));
      I.PB(V(I[k].begin(), I[k].end()));
      Va.PB(V(Va[k].begin(), Va[k].end()));
      Ve.PB(V(Ve[k].begin(), Ve[k].end()));
      aux = graf[v[k]].size();
      I[k+1][v[k]] = 1; Va[k+1][v[k]] = 0;      
      for (i = 0; i < aux; i++) {
        j = graf[v[k]][i]; Ve[k+1][j] = Va[k+1][j] = 0;
      }
      k = k + 1; bandera = 3; break;
      /**********************************************/          
    case 3: // paso 3
      for (i = 0, bandera = 4; i < numv; i++) {
        if (Ve[k][i] == 1) {
          aux = graf[i].size();
          for (j = 0; j < aux; j++)
          if (Va[k][graf[i][j]] == 1) 
            break;
          if (j == aux) { i = numv; bandera = 5; }
        }
      } break;
      /**********************************************/

    case 4: // paso 4
      if (evalua(Ve[k]) == -1 && evalua(Va[k]) == -1) {
        for (int n = 0; n < numv; n++)
          if (I[k][n] == 1) cout<< n << " ";
        cout << endl;
        if (k > kmax) kmax = k;
        bandera = 5;
      }
      else bandera = 2; // ir a paso 2
      break;
      /**********************************************/
    case 5: // paso 5
      k = k - 1; v.pop_back(); I[k].clear();
      I[k].assign(I[k+1].begin(), I[k+1].end());
      I[k][v[k]] = 0; I.pop_back(); Ve.pop_back();
      Va.pop_back(); Ve[k][v[k]] = 1; Va[k][v[k]] = 0;
      if (k == 0)   {
        if (evalua(Va[0]) == -1) return;
        bandera = 2; // ir a paso 2
      }
      else bandera = 3; // ir a paso 3
      break;
    }
  }     
}

int main() {
  U idx, i; stringstream ss; string linea;
  while (cin >> numv) {
    getline(cin, linea);
    for (i = 0; i < numv; i++) { // Lectura del grafo
      // vertices adjacentes al i-esimo vertice
      vector<U> bb; graf.PB(bb);
      getline(cin, linea);
      ss << linea;
      while (ss >> idx) graf[i].PB(idx);
      ss.clear();
    }
    // Llamada al algoritmo
    kmax = 0;
    cout << "Conjuntos independientes: "<< endl;
    if (numv > 0)
      Bron_i_Kerbosch();

    cout << "kmax: " << kmax << endl;
    // Limpieza variables
    for (i = 0; i < numv; i++) graf[i].clear();
    graf.clear();
  }
}
\end{verbatim}

\subsection{Floyd-Warshall (camino mínimo entre todos nodos)}
\begin{verbatim}
// A: matriz n*n de adyacencia con costes
// ausencia de arista representada por un numero grande
for (int k = 0; k < n; k++)
  for (int i = 0; i < n; i++)
    for (int j = 0; j < n; j++)
      A[i][j] = MIN(A[i][j], A[i][k] + A[k][j]);
\end{verbatim}

\subsection{Bellman-Ford (Dijkstra con costes negativos)}
\begin{verbatim}
typedef pair<pair<int,int>,int> P; // par de nodos + coste
int N;                             // numero de nodos
vector<P> v;                       // representacion aristas

int bellmanford(int a, int b) {
  vector<int> d(N, 1000000000);
  d[a] = 0;
  for (int i = 1; i < N; i++)
    for (int j = 0; j < (int)v.SZ; j++)
      if (d[v[j].X.X] < 1000000000 && 
      d[v[j].X.X] + v[j].Y < d[v[j].X.Y])
        d[v[j].X.Y] = d[v[j].X.X] + v[j].Y;
  for (int j = 0; j < (int)v.SZ; j++)
    if (d[v[j].X.X] < 1000000000 &&
    d[v[j].X.X] + v[j].Y < d[v[j].X.Y])
      return -1000000000; // existe ciclo negativo
  return d[b];
}
\end{verbatim}

\subsection{A*}
\begin{verbatim}
int N = 1000;                    // numero de nodos
typedef set<pair<int,int> > S;  // cola de prioridad
vector<pair<int,int> > V[1000];  // lista de adyacencia (coste, vecino)

int heuristic(int from, int to) { // heurística de "from" a "to"
  return 0;
}

int Astar(int from, int to) {

  set<int> open;        // lista abierta de nodos activos
  set<int> closed;      // lista cerrada de nodos ya procesados
  map<int,int> fscore;  // coste real + heurística
  map<int,int> gscore;  // coste real
  S queue;              // cola de prioridad (coste, nodo)
  map<int,int> parent;  // para reconstruir el camino
  
  open.insert( from );
  fscore[ from ] = heuristic( from, to );
  gscore[ from ] = 0;
  queue.insert( make_pair( fscore[ from ], from ) );
  
  while( closed.find( to ) == closed.end() ){
    
    pair<int,int> p = *queue.begin();
    open.erase( p.second );
    closed.insert( p.second );
    queue.erase( queue.begin() );
    
    for( unsigned i = 0; i < V[ p.second ].size(); ++i ){
      
      int neigh = V[ p.second ][i].second;
      int g = gscore[ p.second ] + V[ p.second ][i].first;
      int f = g + heuristic( neigh, to );
      
      if( (open.find( neigh ) == open.end() && closed.find( neigh ) == closed.end() ) || f < fscore[ neigh ] ){
        
        open.erase( neigh );
        queue.erase( make_pair( fscore[ neigh ], neigh ) );
        parent[ neigh ] = p.second;
        fscore[ neigh ] = f;
        gscore[ neigh ] = g;
        open.insert( neigh );
        queue.insert( make_pair( fscore[ neigh ], neigh ) );
        
      }
    }
  }
  
  // reconstruir camino por "parent" si hace falta
  
  int actual = to;
  
  list<int> l;
  list<int>::iterator il;
  
  while( parent[ actual ] != from ){
    l.push_front( actual );
    actual = parent[ actual ];
  }
  
  l.push_front( actual );
  l.push_front( parent[ actual ] );
  
  cout<< "Path = [";
  
  for( il = l.begin(); il != l.end(); il++){
    
    if( *il == to ){
      cout<< *il<<"] with cost ";
      break;
    }
    cout<< *il<<", ";
  }
  return gscore[ to ];
}
int main(){
  N = 3;
  V[0].push_back( make_pair( 1, 1 ) );
  V[1].push_back( make_pair( 2, 2 ) );
  cout<< Astar( 0, 2 ) <<endl;
  return 0;
}
\end{verbatim}

\subsection{[STU] Maximum Bipartite Matching}
\begin{verbatim}
// This code performs maximum bipartite matching.
//
// Running time: O(|E| |V|) -- often much faster in practice
//
//   INPUT: w[i][j] = edge between row node i and column node j
//   OUTPUT: mr[i] = assignment for row node i, -1 if unassigned
//           mc[j] = assignment for column node j, -1 if unassigned
//           function returns number of matches made

#include <vector>

using namespace std;

typedef vector<int> VI;
typedef vector<VI> VVI;

bool FindMatch(int i, const VVI &w, VI &mr, VI &mc, VI &seen) {
  for (int j = 0; j < w[i].size(); j++) {
    if (w[i][j] && !seen[j]) {
      seen[j] = true;
      if (mc[j] < 0 || FindMatch(mc[j], w, mr, mc, seen)) {
        mr[i] = j;
        mc[j] = i;
        return true;
      }
    }
  }
  return false;
}

int BipartiteMatching(const VVI &w, VI &mr, VI &mc) {
  mr = VI(w.size(), -1);
  mc = VI(w[0].size(), -1);
  
  int ct = 0;
  for (int i = 0; i < w.size(); i++) {
    VI seen(w[0].size());
    if (FindMatch(i, w, mr, mc, seen)) ct++;
  }
  return ct;
}
\end{verbatim}

\section{Programación Dinámica}
\subsection{Ejemplo Moneda}
\begin{verbatim}
int N = 8;                // numero de monedas
int m[] = {1,2,5,10,20,50,100,200}; // monedas

int main() {
  int C;         // monto C <= 100000
  cin >> C;
  int A[100001]; // vector de resultados
  A[0] = 0;
  for (int i = 1; i <= C; i++) {
    A[i] = 1000000;
    for (int j = 0; j < N && m[j] <= i; j++)
      A[i] = MIN(A[i], A[i-m[j]] + 1);
  }
  cout << A[C] << endl;
}
\end{verbatim}

\subsection{Ejemplo Mochila}
\begin{verbatim}
int N = 8;        // numero de objetos N <= 1000
int v[] = {1,6,7,1,8,3,7,5}; // valor de objetos
int p[] = {5,3,7,1,8,2,7,3}; // peso de objetos

int main() {
  int C;             // capacidad C <= 1000
  cin >> C;
  int A[1001][1001]; // matriz de resultados
  for (int j = 0; j <= C; j++)
    A[0][j] = 0;
  for (int i = 1; i <= N; i++) {
    A[i][0] = 0;
    for (int j = 1; j <= C; j++) {
      A[i][j] = A[i-1][j];
      if (p[i-1] <= j) {
    int r = A[i-1][j-p[i-1]] + v[i-1];
        A[i][j] = MAX(A[i][j], r);
      }
    }
  }
  cout << A[N][C] << endl;
}
\end{verbatim}

\section{Geometría}
\subsection{Convex Hull}
\begin{verbatim}
typedef int T;     // posiblemente cambiar a double
typedef pair<T,T> P;
T xp(P p, P q, P r) {
  return (q.X-p.X)*(r.Y-p.Y) - (r.X-p.X)*(q.Y-p.Y);
}
struct Vect {
  P p, q;  T dist;
  Vect(P &a, P &b) {
    p = a;  q = b;
    dist = SQ(a.X - b.X) + SQ(a.Y - b.Y);
  }
  bool operator<(const Vect &v) const {
    T t = xp(p, q, v.p);
    return t < 0 || t == 0 && dist < v.dist;
  }
};

vector<P> convexhull(vector<P> v) { // v.SZ >= 2
  sort(v.begin(), v.end());
  vector<Vect> u;
  for (int i = 1; i < (int)v.SZ; i++)
    u.PB(Vect(v[i], v[0]));
  sort(u.begin(), u.end());
  vector<P> w(v.SZ, v[0]);
  int j = 1;  w[1] = u[0].p;
  for (int i = 1; i < (int)u.SZ; i++) {
    T t = xp(w[j-1], w[j], u[i].p);
    for (j--; t < 0 && j > 0; j--)
      t = xp(w[j-1], w[j], u[i].p);
    j += t > 0 ? 2 : 1;
    w[j] = u[i].p;
  }
  w.resize(j+1);
  return w;
}

int main() {
  vector<P> v;
  v.PB(MP(0, 1));
  v.PB(MP(1, 2));
  v.PB(MP(3, 2));
  v.PB(MP(2, 1));
  v.PB(MP(3, 1));
  v.PB(MP(6, 3));
  v.PB(MP(7, 0));
  vector<P> w = convexhull(v);
} // resultado: (0,1) (7,0) (6,3) (1,2)
\end{verbatim}

\subsection{Intersecciones de figuras}
\begin{verbatim}
typedef double T; // dejar incluso cuando puntos son int

class Point {
public:
  T x, y; // punto definido por coordinado (x,y)
  Point() { x = DBL_MAX; y = DBL_MAX; }
  Point(T a, T b) { x = a; y = b; }
  double distance(Point p) { // distancia entre dos puntos
    return sqrt(SQ(p.x - x) + SQ(p.y - y));
  }
  Point operator + (const Point &p)  const { return Point(x+p.x, y+p.y); }
  Point operator - (const Point &p)  const { return Point(x-p.x, y-p.y); }
  Point operator * (double c)     const { return Point(x*c,   y*c  ); }
  Point operator / (double c)     const { return Point(x/c,   y/c  ); }
};

class Line {
public:
  Point p, q; // linea definida por dos puntos DIFERENTES
  Line(Point a, Point b) { p = a; q = b; }
  double distance(Point r) { // distancia de linea a punto
    return r.distance(projection(r));
  }
  // projeccion de punto sobre linea
  Point projection(Point r) {
    if (p.y == q.y)
      return isect(Line(r, Point(r.x, r.y + 1)));
    T z = (q.x - p.x)/(q.y - p.y);
    return isect(Line(r, Point(r.x + 1,r.y - z)));
  }
  Point isect(Line l) { // interseccion entre dos lineas
    if (p.x == q.x) {
      if (l.p.x == l.q.x) // lineas son paralelas
    return Point();
      double a = (l.q.y - l.p.y)/(l.q.x - l.p.x);
      return Point(p.x, l.p.y + a*(p.x - l.p.x));
    }
    else if (l.p.x == l.q.x)
      return l.isect(*this);
    double a1 = (q.y - p.y)/(q.x - p.x);
    double a2 = (l.q.y - l.p.y)/(l.q.x - l.p.x);
    if (a1 == a2) // lines are parallel
      return Point();
    double x = (l.p.y - p.y + p.x*a1 - l.p.x*a2)/(a1-a2);
    return Point(x, p.y + (x - p.x)*a1);
  }
};

class Circle {
public:
  Point p; // punto central del circulo
  T r;     // radio del circulo
  Circle(Point a, T b) { p = a; r = b; }
  bool contains(Point q) { // punto dentro del circulo
    return p.distance(q) <= r;
  }
  // UNA interseccion entre linea and circulo
  Point isect(Line l) {
    Point q = l.projection(p);
    double d = p.distance(q);
    if (d > r) // la linea no intersecta el circulo
      return Point();
    if (l.p.x == l.q.x) {
      double dy = sqrt(SQ(r) - SQ(d));
      return Point(q.x, q.y + dy);
      // otro punto se obtiene por q.y - dy
    }
    double a = (l.q.y - l.p.y)/(l.q.x - l.p.x);
    double x = q.x - p.x;
    double y = q.y - p.y;
    double n = SQ(a) + 1;
    double s = sqrt((SQ(r) - SQ(d))/n);
    double dx = (x + a*y)/n + s;
    // otro punto se obtiene por (x + a*y)/n - s
    return Point(q.x + dx, q.y + a*dx);
  }
};

class Rect {
public:
  Point p, q; // rect definido por dos puntos DIFERENTES
  Rect(Point a, Point b) {
    p = Point(min(a.x, b.x), min(a.y, b.y));
    q = Point(max(a.x, b.x), max(a.y, b.y));
  }
  bool contains(Point r) { // punto dentro del rect
    return r.x>=p.x && r.x<=q.x && r.y>=p.y && r.y<=q.y;
  }

  // UNA interseccion entre rectangulo y linea
  Point isect(Line l) {
    Point r = l.isect(Line(p, Point(q.x, p.y)));
    if (r.x >= p.x && r.x <= q.x)
      return r;
    r = l.isect(Line(p, Point(p.x, q.y)));
    if (r.y >= p.y && r.y <= q.y)
      return r;
    r = l.isect(Line(Point(q.x, p.y), q));
    if (r.y >= p.y && r.y <= q.y)
      return r;
    r = l.isect(Line(Point(p.x, q.y), q));
    if (r.x >= p.x && r.x <= q.x)
      return r;
    return Point();
  }
  // UNA interseccion entre rectangulo y circulo
  Point isect(Circle c) {
    Point r = c.isect(Line(p, Point(q.x, p.y)));
    if (r.x >= p.x && r.x <= q.x)
      return r;
    r = c.isect(Line(p, Point(p.x, q.y)));
    if (r.y >= p.y && r.y <= q.y)
      return r;
    r = c.isect(Line(Point(q.x, p.y), q));
    if (r.y >= p.y && r.y <= q.y)
      return r;
    r = c.isect(Line(Point(p.x, q.y), q));
    if (r.x >= p.x && r.x <= q.x)
      return r;
    return Point();
  }
};

int main() {
  Point p(0, 0);
  Point q(2, 2);
  Point s(0, 2);
  Line l(p, q);
  Circle c(s, 2);
  Rect r(p, q);
  Point t = c.isect(l);
  Point u = r.isect(c);
}
\end{verbatim}
\subsection{[STU] Puntos y polígonos}
Determine if point is in a possibly non-convex polygon (by William
Randolph Franklin); returns 1 for strictly interior points, 0 for
strictly exterior points, and 0 or 1 for the remaining points.
Note that it is possible to convert this into an *exact* test using
integer arithmetic by taking care of the division appropriately
(making sure to deal with signs properly) and then by writing exact
tests for checking point on polygon boundary
\begin{verbatim}
bool PointInPolygon(const vector<PT> &p, PT q) {
  bool c = 0;
  for (int i = 0; i < p.size(); i++){
    int j = (i+1)%p.size();
    if ((p[i].y <= q.y && q.y < p[j].y || 
      p[j].y <= q.y && q.y < p[i].y) &&
      q.x < p[i].x + (p[j].x - p[i].x) * (q.y - p[i].y) / (p[j].y - p[i].y))
      c = !c;
  }
  return c;
}

// determine if point is on the boundary of a polygon
bool PointOnPolygon(const vector<PT> &p, PT q) {
  for (int i = 0; i < p.size(); i++)
    if (dist2(ProjectPointSegment(p[i], p[(i+1)%p.size()], q), q) < EPS)
      return true;
    return false;
}
\end{verbatim}

\section{Secuencias}
\subsection{Ejemplo partición de vector}
\begin{verbatim}
bool IsOdd(int i) {return (i%2==1);}
vector<int>::iterator bound = partition(myvector.begin(), myvector.end(), IsOdd);
\end{verbatim}

\subsection{Búsqueda Binaria}
Utiliza la función \verb-binary\_search- en \verb-algorithm-. Si la tienes que
implementar por algo:
\begin{verbatim}
// devuelve el i mas pequeno tal que t <= v[i]
// si no existe tal i, devuelve v.SZ
template<typename T> int bb(T t, vector<T> &v) {
  int a = 0, b = v.SZ;
  while (a < b) {
    int m = (a + b)/2;
    if (v[m] < t) a = m+1;  else b = m;
  }
  return a;
}
\end{verbatim}
\subsection{Búsqueda Ternaria}
\begin{verbatim}
double E = 0.0000001;
while (1) {
  double dist = R - L;
  if (fabs(dist) < E) break;
  double leftThird = L + dist / 3;
  double rightThird = R - dist / 3;
  if (f(leftThird) < f(rightThird))
    R = rightThird;
  else
    L = leftThird;
}
\end{verbatim}

\subsection{[STU] Suffix Array}
Suffix array construction in $O(L \log^2 L)$ time.
Routine for computing the length of the longest common
prefix of any two suffixes in $O(\log L)$ time.
\begin{verbatim}
// INPUT:   string s
// OUTPUT:  array suffix[] such that suffix[i] = index (from 0 to L-1)
//          of substring s[i...L-1] in the list of sorted suffixes.
//          That is, if we take the inverse of the permutation suffix[],
//          we get the actual suffix array.

#include <vector>
#include <iostream>
#include <string>

using namespace std;

struct SuffixArray {
  const int L;
  string s;
  vector<vector<int> > P;
  vector<pair<pair<int,int>,int> > M;

  SuffixArray(const string &s) : L(s.length()), s(s), P(1, vector<int>(L, 0)), M(L) {
    for (int i = 0; i < L; i++) P[0][i] = int(s[i]);
    for (int skip = 1, level = 1; skip < L; skip *= 2, level++) {
      P.push_back(vector<int>(L, 0));
      for (int i = 0; i < L; i++) 
  M[i] = make_pair(make_pair(P[level-1][i], i + skip < L ? P[level-1][i + skip] : -1000), i);
      sort(M.begin(), M.end());
      for (int i = 0; i < L; i++) 
  P[level][M[i].second] = (i > 0 && M[i].first == M[i-1].first) ? P[level][M[i-1].second] : i;
    }    
  }

  vector<int> GetSuffixArray() { return P.back(); }

  // returns the length of the longest common prefix of s[i...L-1] and s[j...L-1]
  int LongestCommonPrefix(int i, int j) {
    int len = 0;
    if (i == j) return L - i;
    for (int k = P.size() - 1; k >= 0 && i < L && j < L; k--) {
      if (P[k][i] == P[k][j]) {
  i += 1 << k;
  j += 1 << k;
  len += 1 << k;
      }
    }
    return len;
  }
};

int main() {

  // bobocel is the 0'th suffix
  //  obocel is the 5'th suffix
  //   bocel is the 1'st suffix
  //    ocel is the 6'th suffix
  //     cel is the 2'nd suffix
  //      el is the 3'rd suffix
  //       l is the 4'th suffix
  SuffixArray suffix("bobocel");
  vector<int> v = suffix.GetSuffixArray();
  
  // Expected output: 0 5 1 6 2 3 4
  //                  2
  for (int i = 0; i < v.size(); i++) cout << v[i] << " ";
  cout << endl;
  cout << suffix.LongestCommonPrefix(0, 2) << endl;
}
\end{verbatim}

\subsection{[STU] Knuth-Morris-Pratt (búsqueda substrings)}
Searches for the string w in the string s (of length k). Returns the
0-based index of the first match (k if no match is found). Algorithm
runs in $O(k)$ time.
\begin{verbatim}
typedef vector<int> VI;

void buildTable(string& w, VI& t) {
  t = VI(w.length());  
  int i = 2, j = 0;
  t[0] = -1; t[1] = 0;
  
  while(i < w.length()) {
    if(w[i-1] == w[j]) { t[i] = j+1; i++; j++; }
    else if(j > 0) j = t[j];
    else { t[i] = 0; i++; }
  }
}

int KMP(string& s, string& w) {
  int m = 0, i = 0;
  VI t;
  
  buildTable(w, t);  
  while(m+i < s.length()) {
    if(w[i] == s[m+i]) {
      i++;
      if(i == w.length()) return m;
    }
    else {
      m += i-t[i];
      if(i > 0) i = t[i];
    }
  }  
}
\end{verbatim}
\end{document}
